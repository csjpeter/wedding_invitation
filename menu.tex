%
\documentclass[10pt,twoside,final]{article}
\usepackage[dvips=false,pdftex=false,vtex=false,truedimen,portrait]{geometry}
\geometry{
	paperwidth=320truemm,
	paperheight=225truemm,
	top=0truecm,
	bottom=0truecm,
	left=0truecm,
	right=0truecm,
	nohead
}
%
%	font setting
\usepackage[T1]{fontenc}
\usepackage{ucs}
\usepackage[utf8x,utf8]{inputenc}
%\usepackage[utf8ttf]{inputenc}
\usepackage{ttfucs}
%
%
\usepackage[hungarian]{babel}	% division of words / elválasztások
\selectlanguage{hungarian}	% language settings / nyelvi beállítások
%
%
\usepackage{color}
%
%
\usepackage{textcomp}
\usepackage{wallpaper}
\usepackage{rotating}
\usepackage[absolute,showboxes]{textpos}
%\usepackage[colorgrid,texcoord]{eso-pic}
%
%
%
\newcount{\paperWidth}\paperWidth\paperwidth%
\divide\paperWidth186467%
\newcount{\paperHeight}\paperHeight\paperheight%
\divide\paperHeight186467%
\newcount{\vertMargin}\vertMargin5%
\newcount{\horizMargin}\horizMargin5%
\newcount{\columnTextMargin}\columnTextMargin16%
\newcount{\columnWidth}\columnWidth95%
%
\newcount{\columnTextWidth}\columnTextWidth\columnWidth%
\advance\columnTextWidth-\columnTextMargin
%
\newcount{\usedPaperWidth}\usedPaperWidth\paperWidth%
\advance\usedPaperWidth-\horizMargin
\advance\usedPaperWidth-\horizMargin
\newcount{\usedPaperHeight}\usedPaperHeight\paperHeight%
\advance\usedPaperHeight-\vertMargin
\advance\usedPaperHeight-\vertMargin
%
\newcount{\topMargin}\topMargin\vertMargin%
\newcount{\bottomMargin}\bottomMargin\topMargin%
\advance\bottomMargin\usedPaperHeight
\newcount{\leftMargin}\leftMargin\horizMargin%
\newcount{\rightMargin}\rightMargin\horizMargin%
\advance\rightMargin\usedPaperWidth
%
%adjust the TPHorizModule and TPHorizModule units to the displayed mm %grid
\TPGrid{\paperWidth}{\paperHeight}
\setlength{\TPboxrulesize}{0 pt}
%puts a graphic at the absolute position described by the grid
%#1 x, #2 y, #3 width, #4 height, #5 graphic
\newcommand\putpic[5]{%
	\noindent
	\begin{textblock}{#3}(#1,#2)
		\noindent
		\includegraphics[width=#3\TPHorizModule,height=#4\TPVertModule]{#5}
	\end{textblock}
}
%
%
%
\DeclareTruetypeFont{pamplona}{pamplona}
%
%\definecolor{decorcolor}{rgb}{0.25,0.125,0}
\definecolor{textcolor}{rgb}{0,0,0}
\definecolor{browncolor}{rgb}{0.125,0.0625,0}
\definecolor{background}{rgb}{0.99,0.99,0.85}
%
%\definecolor{decorcolor}{rgb}{0,0,0}
%\definecolor{textcolor}{rgb}{0,0,0}
%
\newcommand{\fttitle}{
	\renewcommand{\baselinestretch}{1}
	\inputencoding{utf8x}
	\usefont{T1}{pamplona}{m}{n}
	\fontsize{14mm}{14mm}\selectfont
	\color{textcolor}
}
\newcommand{\ftsubtitle}{
	\renewcommand{\baselinestretch}{1}
	\inputencoding{utf8x}
	\usefont{T1}{pamplona}{m}{n}
	\Huge
	\fontsize{10mm}{10mm}\selectfont
	\color{textcolor}
}
\newcommand{\ftnormal}{
	\renewcommand{\baselinestretch}{1.2}
	\inputencoding{utf8}
	\usefont{T1}{pnc}{m}{it}
	\normalsize
	\large
%	\renewcommand{\baselinestretch}{1.5}
%	\inputencoding{utf8}
%	\usefont{T1}{pnc}{m}{it}
%	\normalsize
	\color{textcolor}
}
\newcommand{\ftsmall}{
	\renewcommand{\baselinestretch}{1.2}
	\inputencoding{utf8}
	\usefont{T1}{pnc}{m}{it}
	\normalsize
%	\large
%	\renewcommand{\baselinestretch}{1.5}
%	\inputencoding{utf8}
%	\usefont{T1}{pnc}{m}{it}
%	\normalsize
	\color{textcolor}
}
%
\newcommand{\cutmarkup}[2]{
	\noindent
	\begin{textblock}{0}(#1,#2)
	\noindent
	\setlength{\unitlength}{3mm}
	\begin{picture}(2, 2)
		\put(-1, 2){\line(1, 0){2}} % horiz
		\put(0, 2){\line(0, 1){1}} % vert
%		\put(0, 2){\circle{0.1}}\put(0, 2){\circle{2}} % origo
	\end{picture}
	\end{textblock}
}
%
\newcommand{\cutmarkdown}[2]{
	\noindent
	\begin{textblock}{0}(#1,#2)
	\noindent
	\setlength{\unitlength}{3mm}
	\begin{picture}(2, 2)
		\put(-1, 2){\line(1, 0){2}} % horiz
		\put(0, 1){\line(0, 1){1}} % vert
%		\put(0, 2){\circle{0.1}}\put(0, 2){\circle{2}} % origo
	\end{picture}
	\end{textblock}
}
%
\newcommand{\cutmark}[2]{
	\noindent
	\begin{textblock}{0}(#1,#2)
	\noindent
	\setlength{\unitlength}{3mm}
	\begin{picture}(2, 2)
		\put(-1, 2){\line(1, 0){2}} % horiz
		\put(0, 1){\line(0, 1){2}} % vert
%		\put(0, 2){\circle{0.1}}\put(0, 2){\circle{2}} % origo
	\end{picture}
	\end{textblock}
}
%
\newcommand{\helper}{
	\noindent
%	\begin{textblock}{0}(\paperWidth,\paperHeight)
	\begin{textblock}{0}(0,0)
	\noindent
	\setlength{\unitlength}{1mm}
	\begin{picture}(\paperWidth, \paperHeight)
		\put(\leftMargin, \topMargin){\line(1, 0){\usedPaperWidth}} % horiz
		\put(\leftMargin, \bottomMargin){\line(1, 0){\usedPaperWidth}} % horiz
		%
		\var\leftMargin
		\put(\var, \topMargin){\line(0, 1){\usedPaperHeight}} % horiz
		\advance\var\columnWidth
		\put(\var, \topMargin){\line(0, 1){\usedPaperHeight}} % horiz
		\advance\var\columnWidth
		\put(\var, \topMargin){\line(0, 1){\usedPaperHeight}} % horiz
		\advance\var\columnWidth
		\put(\var, \topMargin){\line(0, 1){\usedPaperHeight}} % horiz
	\end{picture}
	\end{textblock}
}
%
%
%
\newcount{\var}
%
\begin{document}
\pagestyle{empty}
%\pagecolor{background}
%\helper
%
%
%
\var\leftMargin
\cutmarkup{\var}{\topMargin}
\cutmarkdown{\var}{\bottomMargin}
\begin{textblock}{\columnWidth}(\var,\topMargin)
	\noindent
	\begin{center}
	\begin{minipage}[t]{\columnTextWidth mm}
	\color{textcolor}
	%
	\vspace{15mm}
	\begin{center}\fttitle{Lakodalmas menü / Dishes}\end{center}
	\vspace{10mm}
	%
	\ftsubtitle{Vacsora}
	\begin{list}{}{\ftnormal\parsep0pt\itemsep0pt}
	\item Tanyasi tyúkhúsleves
	\item Levesben főtt zsenge tyúkhús
		\begin{list}{}{\ftsmall\parsep0pt\itemsep0pt}
		\item paradicsom és fokhagymamártással
		\end{list}
	\item Marhapörkölt galuskával
		\begin{list}{}{\ftsmall\parsep0pt\itemsep0pt}
		\item (svédasztalon)
		\end{list}
	\item Lakodalmas tál
		\begin{list}{}{\ftnormal\parsep0pt\itemsep0pt}
		\item Borzas pulykamell rántva
		\item Fokhagymás tarjapecsenye
		\item Fűszeres töltött csirkecomb
		\item Rántott gomba és karfiol
		\item
		\item Petrezselymes burgonya
		\item Rizi--bizi
		\item Mandulás krokett
		\item
		\item Vegyes saláta 
		\end{list}
	\item Sütemények
	\end{list}
	%
	\vspace{5mm}
	\ftsubtitle{Éjféli vacsora}
	\begin{list}{}{\ftnormal\parsep0pt\itemsep0pt}
	\item Kolozsvári töltött káposzta
	\item Hideg sültek franciasalátával
	\end{list}
	%
	\end{minipage}
	\end{center}
\end{textblock}
%
%
%
\advance\var\columnWidth
\cutmarkup{\var}{\topMargin}
\cutmarkdown{\var}{\bottomMargin}
\begin{textblock}{\columnWidth}(\var,\topMargin)
	\noindent
	\begin{center}
	\begin{minipage}[t]{\columnTextWidth mm}
	\color{textcolor}
	%
	\vspace{15mm}
	\begin{center}\fttitle{Itallap / Drinks}\end{center}
	\vspace{5mm}
	%
	\ftsubtitle{Üdítő}
		\begin{list}{}{\ftnormal\parsep0pt\itemsep0pt}
		\item Gyümölcslevek: almalé, narancslé
		\item Szénsavas Cola, Fanta, Tonic
		\item Ásványvíz (mentes és szénsavas)
		\end{list}
	\ftsubtitle{Alkoholos italok}
		\begin{list}{}{\ftnormal\parsep0pt\itemsep0pt}
		\item Sör
			\begin{list}{}{\ftsmall\parsep0pt\itemsep0pt}
			\item Gösser világos és barna sör
			\item Gösser alkoholmentes sör
%			\item Edelweiss búzasör
			\end{list}
		\item Bor (Hetényi Pincészet)
			\begin{list}{}{\ftsmall\parsep0pt\itemsep0pt}
			\item Vörös Cuvée száraz
			\item Kékfrankos Rosé száraz
			\item Chardonnay száraz
			\item Édes vörös (svédasztalon)
			\end{list}
		\item Röviditalok
			\begin{list}{}{\ftsmall\parsep0pt\itemsep0pt}
			\item Jägermeister
			\item Kecskeméti barackpálinka
			\item Martini Bianco
			\item Johnny Walker Whisky
			\item Caroline's krémlikőr
			\end{list}
		\end{list}
	\ftsubtitle{Kávék}
		\begin{list}{}{\ftnormal\parsep0pt\itemsep0pt}
		\item Hosszúkávé
		\item Tejszínhabos kávé
		\end{list}
	%
	\end{minipage}
	\end{center}
\end{textblock}
%
%
%
\advance\var\columnWidth
\cutmarkup{\var}{\topMargin}
\cutmarkdown{\var}{\bottomMargin}
\begin{textblock}{\columnWidth}(\var,\topMargin)
	\noindent
	\begin{center}
	\begin{minipage}[t]{\columnTextWidth mm}
	\color{textcolor}
	%
	\vspace{15mm}
	\begin{center}\fttitle{Dear guests, / Kedves Vendégeink!}\end{center}
	\vspace{10mm}
	%
	\ftnormal
	Welcome text / Üdvüzlő szövegelés
	\\%
	\vspace{20mm}\\
	More text / További szövegelés
	\\%
	{\flushright Signature / Aláírás, stb \vspace{5mm}\\}
	{\flushright \ftsubtitle Bride and Groom / Menyasszony és Vőlegény\\}
	\end{minipage}
	\end{center}
\end{textblock}
%
\advance\var\columnWidth
\cutmarkup{\var}{\topMargin}
\cutmarkdown{\var}{\bottomMargin}
%
\cutmarkup{\rightMargin}{\topMargin}
\cutmarkdown{\rightMargin}{\bottomMargin}
%
%
%
\end{document}
%
%
%
%\tiny
%\scriptsize
%\footnotesize
%\small
%\normalsize
%\large
%\Large
%\LARGE
%\huge
%\Huge 
